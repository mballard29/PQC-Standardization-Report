\graphicspath{{Images/}}

\section{Quantum Communication and Cryptography} \label{q_crypto}
If cryptography is the ''art of secret writing'' \cite{netsec}, then \textbf{\gls{q_crypto}} is cryptography that takes advantage of the unique properties of quantum computers to provide \gls{confidentiality}, \gls{integrity}, \gls{non-repudiation}, \gls{authenticity}, and \gls{availability} (see the \gls{cia} in the glossary). If classical computers use mathematical operations to mangle and obscure data, quantum computers use quantum mechanics and physics to do the same. In this way, quantum computers' unique properties allow quantum cryptography to have built in security in ways not possible and more reliable than for cryptography done with classical computers. For example, the no-cloning theorem makes error checking and correction difficult, but it also makes quantum communication more secure.  More generally, \gls{crypto} is:

\begin{quote}
    ``the ability to send information between participants in a way that prevents others from reading it.'' \cite{netsec}
\end{quote}

 To illustrate how cryptography can be made more secure using quantum computers rather than classical computers, we will introduce two common attack vectors that affect both classical and quantum cryptography, then we will use a basic quantum communication protocol to show how quantum computing can be more effective at defending against these vectors.

    \subsection{Common Attack Vectors}
    An \textbf{\gls{attackvector}} is:
    \begin{quote}
        ``the method or combination of methods that cybercriminals use to breach or infiltrate a victim's network'' \cite{attackvectors}
    \end{quote}
    Attack vectors may be individual means of accessing a network or it may be a set of tools and techniques combined to do so. Many attack vectors may be used as introductory or intermediary steps of getting into a system which means they may provide a means of compromising more security objectives than the immediate attack vector itself does.  
    
    \subsubsection{\textbf{\Gls{eavesdropping}}}
    Eavesdropping describes an attack where an adversary may intercept communication. It is a passive attack meaning that the information stream is not modified in any way. However, such an attack does violate the security objective of confidentiality.
    
    \subsubsection{Man-In-The-Middle (\textbf{\gls{mitm}})}
    A man-in-the-middle attack (or meddler-in-the-middle attack) is the active form of eavesdropping wherein an adversary may alter the information stream. They may ``transmit their own messages, replay old messages, modify messages in transit, or delete or delay selected messages in transit'' \cite{netsec}. Because this attack involves an adversary reading and altering information not meant for them, it violates the security objectives of confidentiality and integrity.

    \subsection{Quantum Key Distribution (\textbf{\gls{qkd}})}
    The quantum key distribution method is a secure cryptographic communication protocol which allows participants to generate a shared key. This shared key may be used to privately negotiate an encryption scheme or as a one-time pad (see glossary \textbf{\gls{otp}}). Similar to the previously mentioned BB84, it is based on two principles: the \gls{uncertainty} (see glossary) and the quantum property of entanglement. It works in this way:
    
    \begin{table}[h!]
        \begin{tabular}{cl}{Given:} \\
             $A \leftarrow $ & the sender of some secret information\\
             $B \leftarrow $ & the receiver of some secret information \\
        \end{tabular}
        \label{tab:given_vars}
    \end{table}

    \begin{algorithm}
        \caption{The QKD Procedure is as follows:}
        \label{alg:qkd}
        \begin{algorithmic}[1]
            \Procedure{QKD}{$A$, $B$}
                \State $bits_{sent} \gets \Call{SendBits}{bits_A, Filter_A}$
                \State $Filter_B \gets \Call{rand}{Filters}$
                \Comment{where $Filters$ is a set of filters}
                \State $bits_B \gets \Call{ReceiveBits}{bits_sent, Filter_B}$
                \State $key_{AB} \gets \Call{Compare}{Filter_A, Filter_B}$ \label{compare}
            \EndProcedure
        \end{algorithmic}
    \end{algorithm}

    In this procedure, $A$ sends some bits through a polarizing filter (see Appendix \ref{note:polarize}) to $B$. Because $B$ doesn't know which polarizing filters $A$ used, they use a random set of filters to receive these bits through. Once A has finished sending these bits and B has finished receiving them, they compare the polarizing filters they used. They keep the bits sent/received that match and discard those that do not. This set of matching bits becomes the shared key. Now, how does this protect against eavesdropping or MITM attacks?

    \subsubsection{Protecting Against Eavesdropping and MITM Attacks}
    An adversary could try to listen in on the communication between $A$ and $B$, but reading a state destroys the superposition information which effectively changes the state. An adversary could try to use their own polarizing filter to read and/or pass on the information transmitted, but without being to compare (step \ref{compare}), any information they collect is somewhat useless. Even better than the fact that an outsider is not able to effectively listen in or manipulate information communicated using this protocol is the way that doing so alters the information stream in a way that very quickly alerts the communicating parties that there is some interfering force that is not supposed to be there. This ability to protect against and detect any attempted meddling means that communicating parties can quickly and effectively respond (i.e. starting over), and they can rest assured that future communication will be secure from having been breached at this initial stage. 