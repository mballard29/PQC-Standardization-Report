\begin{titlepage}
    \begin{center}
    {\fontsize{40}{48}\selectfont \bfseries Cryptography in a Post-Quantum World} 
    \\\vspace{20pt}
    {\LARGE Off to the Races} \\
    \vspace{20pt}
    \textbf{Mason Ballard}
    \vspace{8pt}
    \\ CNT 5412, Fall 2023
    \end{center}

    \bigskip
    \begin{abstract}
       Quantum computing is advancing. Promising speed and power, it has great potential to advance the world we live in while also introducing some new threats. The modern world has become increasing intertwined with the digital world including means of communication, employment, education, public health, commerce, and more. One of the main threats of post-quantum computing is its ability to break current public-key cryptography which in turn could endanger the private information of a great number of people and organizations, important critical infrastructure sectors, and critical infrastructure functions. Though NIST’s updated standards are not expected to be completed and released until 2024, NIST currently has been conducting conferences and calls for papers, choosing algorithms to be standardized and recommended for use as well as collaborating with the NSA and DHS to solicit ideas and research, document decisions, and disseminate recommendations for the public and private sector. The first part of this paper will discuss what quantum computing is with particular attention it’s power within cryptography. Second, we will go over the threats that such increased power will have. Third, we will go over the standards, recommendations, and guidelines in place currently. Because the standardization process is still ongoing we will discuss issues still present to address, the most recent updates gained from research or conferences, and the predicted next steps in this process and it’s forward-looking impact on our world.
    \end{abstract}
\end{titlepage}